\documentclass{article}
\usepackage{amsmath}

\title{opdracht 2 wavelet}
\author{Isidoor Pinillo Esquivel}
\date{\today}

\begin{document}

\maketitle

\section{Eigenschap 7.2}

\subsection{Te Bewijzen}

Admissibility condition:

$2\pi\int_{\mathbb{R}^*}{\frac{|\widehat{\psi}(a)|^{2}}{|a|}}da=:C_{\psi}<\infty.$

$\forall \psi \in L^{2}, ||\psi|| =1 \text{ en } t \psi \in L^{1}: \text{Admissibility condition} \Leftrightarrow $

$\int_{-\infty}^{\infty}\psi(t)\,dt=0\qquad\mathrm{resp.}\qquad\widehat{\psi}(0)=0\\ .
$

\subsection{Bewijs}

Door dat $t \psi \in L^{1}$ is $\widehat{\psi}$ continu afleidbaar (Lemma 2.17).

De middelwaarden stelling geeft: $\forall |z|\le 1,\exists |c(z)|\le 1 :$

\begin{align*}
    \widehat{\psi}(z)   & = \widehat{\psi}(0)+\widehat{\psi}'(c) z \Rightarrow  \\
    |\widehat{\psi}(z)| & \ge |\widehat{\psi}(0)| - |\widehat{\psi}'(c(z)) ||z| \\
    |\widehat{\psi}(z)| & \le |\widehat{\psi}(0)| + |\widehat{\psi}'(c(z)) ||z|
\end{align*}

Opmerking omdat $\widehat{\psi}'$ continu is op een compact interval is het ook eindig en is $|\widehat{\psi}'(c(z)) |$ ook eindig.

\subsubsection{Bewijs ($\Rightarrow$)}

\begin{align*}
    \infty & > 2 \pi \int_{\mathbb{R}^*} \frac{|\widehat{\psi}(a)|^{2} }{|a|} da                                                                                     \\
           & > 2 \pi \int_{-1}^{1} \frac{|\widehat{\psi}(0)|^{2} - 2 |a| |\widehat{\psi}'(c(a))||\widehat{\psi}(0)| + |a|^{2} |\widehat{\psi}'(c(a))|^{2} }{|a|} da.
\end{align*}

De $2$ laatste termen zijn eindig dus irrelevant voor de ongelijkheid:

\begin{align*}
    \infty            & > \int_{-1}^{1} \frac{|\widehat{\psi}(0)|^{2}}{|a|} da \Leftrightarrow \\
    \widehat{\psi}(0) & =0
    .
\end{align*}

\subsubsection{Bewijs ($\Leftarrow$)}

Krijg dit niet juist gerenderd in github, ik gebruik hier de andere ongelijkheid en de veronderstelling.

Stel dat $\widehat{\psi}(0) =0$.

\begin{align*}
     & 2\pi \int_{\mathbb{R}^*} \frac{|\widehat{\psi}(a)|^{2} }{|a|} da                                                                                    \\
     & \le 2 \pi \int_{\mathbb{R}^* \setminus [-1,1]} \frac{|\widehat{\psi}(a)|^{2} }{|a|} da +2 \pi \int_{-1}^{1} \frac{|\widehat{\psi}(a)|^{2} }{|a|} da \\
     & \le 2 \pi \int_{\mathbb{R}^* \setminus [-1,1]} |\widehat{\psi}(a)|^{2} da +2 \pi \int_{-1}^{1} \frac{|\widehat{\psi}(a)|^{2} }{|a|} da              \\
     & \le 2 \pi \int_{\mathbb{R}^* } |\widehat{\psi}(a)|^{2} da +2 \pi \int_{-1}^{1} \frac{|\widehat{\psi}(a)|^{2} }{|a|} da                              \\
     & \le 2 \pi \int_{\mathbb{R}^* } |\psi(a)|^{2} da +2 \pi \int_{-1}^{1} \frac{|\widehat{\psi}(a)|^{2} }{|a|} da                                        \\
     & \le \text{eindig} +2 \pi \int_{-1}^{1} \frac{|\widehat{\psi}(a)|^{2} }{|a|} da                                                                      \\
     & \le \text{eindig} +2 \pi \int_{-1}^{1} \frac{|\widehat{\psi}'(c(a))|^{2} |a|^{2} }{|a|} da                                                          \\
     & \le \text{eindig} +2 \pi \int_{-1}^{1} |\widehat{\psi}'(c(a))|^{2} |a| da                                                                           \\
     & \le \text{eindig} + \text{eindig}                                                                                                                   \\
     & < \infty.
\end{align*}

\section{Voorbeeld 9.2}

 (b) en (c) volgen bijna direct uit de definitie, bij (c) is ONB triviaal.

De inclusies volgt uit de schalingsvergelijking (Stelling 9.4), $h_{0}=h_{1} = \frac{1}{\sqrt{2}}$.

Volledigheid en separatie kunnen bewezen worden zoals in (Stelling 9.8) ze volgen er ook uit:

$|\phi(t)| \le \frac{2}{1 + t^{2}} $, $| \int \phi(t)dt | = 1 $.

\section{Voorbeeld 9.15}

\begin{align*}
     & \int_{0}^{0.5} e^{itz} dt - \int_{0.5}^{1} e^{itz} dt                                 \\
     & = \frac{1}{iz} \left( 2 e^{\frac{iz}{2}} - 1 - e^{iz} \right)                         \\
     & = \frac{e^{\frac{iz}{2}}}{iz} \left( 2 - e^{\frac{-iz}{2}} - e^{\frac{iz}{2}} \right) \\
     & = \frac{e^{\frac{iz}{2}}}{iz} \left( 2 - 2 \cos \left( \frac{z}{2} \right) \right)    \\
     & = \frac{4 e^{\frac{iz}{2}}}{iz} \sin^{2} \left( \frac{z}{4} \right)
    .
\end{align*}

\end{document}